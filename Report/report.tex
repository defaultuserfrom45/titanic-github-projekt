\documentclass[12pt,a4paper]{article}

\usepackage[utf8]{inputenc}
\usepackage[T1]{fontenc}
\usepackage[ngerman]{babel}
\usepackage{graphicx}
\usepackage{booktabs}
\usepackage{float}
\usepackage{geometry}
\usepackage{setspace}

\geometry{left=3cm,right=3cm,top=3cm,bottom=3cm}
\onehalfspacing
\setlength{\parindent}{0pt}

\title{Analyse des Titanic-Datensatzes}
\author{Rit Alija \and Yasin Sabanoglu}
\date{\today}

\begin{document}
\maketitle

\section{Einleitung}

Der Titanic-Datensatz ist ein häufig verwendeter Beispieldatensatz zur Einführung in
Datenanalyse und Statistik. Er enthält Informationen über Passagiere der Titanic,
unter anderem zu Geschlecht, Alter, Ticketklasse, Ticketpreis und Überlebensstatus.

Ziel dieser Arbeit ist es, den Datensatz aufzubereiten, deskriptiv zu analysieren
und erste Zusammenhänge zwischen ausgewählten Variablen zu untersuchen.

\section{Datenaufbereitung}

Die Rohdaten wurden zunächst bereinigt und für die Analyse vorbereitet.
Dabei wurden unter anderem folgende Schritte durchgeführt:

\begin{itemize}
  \item Extraktion und Vereinheitlichung der Anrede aus dem Namen
  \item Umkodierung geeigneter Variablen in Faktoren
  \item Imputation fehlender Alterswerte anhand des Medianalters pro Anrede
  \item Aufbereitung der Kabineninformation (Deck und Seite)
  \item Entfernen nicht benötigter Variablen
\end{itemize}

Der bereinigte Datensatz wurde anschließend als neue CSV-Datei gespeichert und
diente als Grundlage für alle weiteren Analysen.

\section{Deskriptive Analyse}

\subsection{Metrische Variablen}

Tabelle~\ref{tab:age} zeigt die deskriptiven Kennzahlen für das Alter der Passagiere.

\begin{table}[H]
\centering
\caption{Deskriptive Statistik des Alters}
\label{tab:age}
\begin{tabular}{lr}
\toprule
Kennzahl & Wert \\
\midrule
Anzahl & 891 \\
Mittelwert & 29.7 \\
Median & 28.0 \\
Standardabweichung & 14.5 \\
Minimum & 0.4 \\
Maximum & 80.0 \\
\bottomrule
\end{tabular}
\end{table}

Tabelle~\ref{tab:fare} zeigt die deskriptiven Kennzahlen für den Ticketpreis.

\begin{table}[H]
\centering
\caption{Deskriptive Statistik des Ticketpreises}
\label{tab:fare}
\begin{tabular}{lr}
\toprule
Kennzahl & Wert \\
\midrule
Anzahl & 891 \\
Mittelwert & 32.2 \\
Median & 14.5 \\
Standardabweichung & 49.7 \\
Minimum & 0.0 \\
Maximum & 512.3 \\
\bottomrule
\end{tabular}
\end{table}

\subsection{Kategoriale Variablen}

Tabelle~\ref{tab:survived} zeigt die Verteilung des Überlebensstatus.

\begin{table}[H]
\centering
\caption{Überlebensstatus der Passagiere}
\label{tab:survived}
\begin{tabular}{lrr}
\toprule
Status & Häufigkeit & Anteil \\
\midrule
Ja & 342 & 0.384 \\
Nein & 549 & 0.616 \\
\bottomrule
\end{tabular}
\end{table}

\section{Bivariate Analyse}

Abbildung~\ref{fig:pclass} zeigt die Überlebenshäufigkeit in Abhängigkeit von der Ticketklasse.

\begin{figure}[H]
\centering
\includegraphics[width=0.8\textwidth]{figures/bar_survived_by_pclass.png}
\caption{Überleben nach Ticketklasse}
\label{fig:pclass}
\end{figure}

Abbildung~\ref{fig:sex} zeigt den Zusammenhang zwischen Geschlecht und Überlebensstatus.

\begin{figure}[H]
\centering
\includegraphics[width=0.8\textwidth]{figures/bar_survived_by_sex.png}
\caption{Überleben nach Geschlecht}
\label{fig:sex}
\end{figure}

Abbildung~\ref{fig:embarked} zeigt den Überlebensstatus nach Einschiffungshafen.

\begin{figure}[H]
\centering
\includegraphics[width=0.8\textwidth]{figures/bar_survived_by_embarked.png}
\caption{Überleben nach Einschiffungshafen}
\label{fig:embarked}
\end{figure}

Abbildung~\ref{fig:agebox} zeigt die Altersverteilung getrennt nach Überlebensstatus.

\begin{figure}[H]
\centering
\includegraphics[width=0.8\textwidth]{figures/box_age_by_survived.png}
\caption{Alter nach Überlebensstatus}
\label{fig:agebox}
\end{figure}

Abbildung~\ref{fig:farebox} zeigt die Verteilung der Ticketpreise nach Überlebensstatus.

\begin{figure}[H]
\centering
\includegraphics[width=0.8\textwidth]{figures/box_fare_by_survived.png}
\caption{Ticketpreis nach Überlebensstatus}
\label{fig:farebox}
\end{figure}

Abbildung~\ref{fig:triple} zeigt die gemeinsame Verteilung von Überlebensstatus,
Geschlecht und Ticketklasse.

\begin{figure}[H]
\centering
\includegraphics[width=0.8\textwidth]{figures/viz_survived_sex_pclass.png}
\caption{Überleben nach Geschlecht und Ticketklasse}
\label{fig:triple}
\end{figure}

\section{Diskussion}

Die Ergebnisse zeigen deutliche Unterschiede in den Überlebenswahrscheinlichkeiten
zwischen verschiedenen Gruppen. Insbesondere weibliche Passagiere sowie Passagiere
höherer Klassen hatten eine deutlich höhere Überlebenschance.
Auch der Ticketpreis unterscheidet sich klar zwischen Überlebenden und Nicht-Überlebenden.

\section{Fazit}

In dieser Arbeit wurde der Titanic-Datensatz erfolgreich aufbereitet und analysiert.
Die Ergebnisse verdeutlichen bekannte Muster und zeigen den Nutzen strukturierter
Datenanalyse anhand eines realitätsnahen Beispiels.

\end{document}
